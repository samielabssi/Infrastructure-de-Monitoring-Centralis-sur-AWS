\documentclass[12pt,a4paper]{article}
\usepackage[utf8]{inputenc}
\usepackage[T1]{fontenc}
\usepackage[french]{babel}
\usepackage{graphicx}
\usepackage{hyperref}
\usepackage{geometry}
\usepackage{setspace}
\usepackage{titlesec}
\usepackage{lmodern}
\usepackage{xcolor}
\usepackage{listings}
\usepackage{caption}
\usepackage{subcaption}
\usepackage{fancyhdr}
\usepackage{enumitem}
\usepackage{float}

% Configuration des marges
\geometry{top=2.5cm, bottom=2.5cm, left=2.5cm, right=2.5cm}

% Configuration des titres
\titleformat{\section}{\Large\bfseries\color{blue}}{ \thesection}{1em}{}
\titleformat{\subsection}{\large\bfseries\color{black}}{ \thesubsection}{1em}{}

% Configuration des en-têtes et pieds de page
\pagestyle{fancy}
\fancyhf{}
\fancyhead[L]{TP Monitoring AWS}
\fancyhead[R]{\leftmark}
\fancyfoot[C]{\thepage}
\renewcommand{\headrulewidth}{0.4pt}
\renewcommand{\footrulewidth}{0.4pt}

% Configuration pour les codes sources
\lstdefinestyle{mystyle}{
    backgroundcolor=\color{white},
    basicstyle=\ttfamily\small,
    breakatwhitespace=false,
    breaklines=true,
    captionpos=b,
    keepspaces=true,
    showspaces=false,
    showstringspaces=false,
    showtabs=false,
    tabsize=2,
    frame=single,
    numbers=left,
    numberstyle=\tiny,
    language=bash
}
\lstset{style=mystyle}

% Métadonnées du document
\title{\textbf{Mise en œuvre d'une infrastructure cloud de supervision centralisée sous AWS:\\Déploiement de Zabbix conteneurisé pour le monitoring d'un parc hybride (Linux \& Windows)}}
\author{}
\date{}

\begin{document}

% Page de garde
\begin{titlepage}
    \begin{center}
        % Logo de l'établissement (à remplacer par votre logo)
        \includegraphics[width=0.3\textwidth]{logo_etablissement.png}\\[1.5cm]
        
        {\LARGE \textbf{Rapport de Travail Pratique}}\\[0.5cm]
        
        {\Huge \textbf{Infrastructure de Monitoring}}\\[0.3cm]
        {\Huge \textbf{Centralisé sur AWS}}\\[1cm]
        
        {\Large \textbf{Titre:} Mise en œuvre d'une infrastructure cloud de supervision centralisée sous AWS}\\[1.5cm]
        
        \begin{minipage}[t]{0.4\textwidth}
            \begin{flushleft}
                \textbf{Réalisé par:}\\
                El Abssi Sami\\
            \end{flushleft}
        \end{minipage}
        \begin{minipage}[t]{0.4\textwidth}
            \begin{flushright}
                \textbf{Encadré par:}\\
                Prof. Azeddine KHIAT
            \end{flushright}
        \end{minipage}\\[2cm]
        
        \textbf{Filière:} Ingénieur informatique et intelligence artificielle\\[1cm]
        
        \textbf{Année universitaire: 2025/2026}\\[2cm]
        
        \textbf{Dépôt GitHub:}\\
        \href{https://github.com/samielabssi/Infrastructure-de-Monitoring-Centralis-sur-AWS}{https://github.com/samielabssi/Infrastructure-de-Monitoring-Centralis-sur-AWS}
    \end{center}
\end{titlepage}

% Table des matières
\newpage
\tableofcontents
\newpage

% Introduction
\section{Introduction}

\subsection{Présentation du projet}
Ce travail pratique a pour objectif de déployer une infrastructure de monitoring centralisée sur Amazon Web Services (AWS) en utilisant Zabbix conteneurisé avec Docker. L'infrastructure vise à surveiller un parc hybride composé de machines Linux et Windows, démontrant ainsi la flexibilité et la puissance de Zabbix dans un environnement cloud.

\subsection{Outils utilisés}
\begin{itemize}[leftmargin=*]
    \item \textbf{AWS (Amazon Web Services)}: Plateforme cloud utilisée pour héberger toute l'infrastructure (instances EC2, VPC, groupes de sécurité).
    \item \textbf{Docker}: Solution de conteneurisation permettant de déployer facilement Zabbix et ses composants (serveur, base de données, interface web).
    \item \textbf{Zabbix}: Solution de monitoring open-source offrant une surveillance complète des ressources système et réseau.
\end{itemize}

\subsection{Architecture globale}
L'infrastructure déployée comprend:
\begin{itemize}
    \item Un VPC avec un sous-réseau public
    \item Trois instances EC2: serveur Zabbix, client Linux, client Windows
    \item Des groupes de sécurité pour contrôler les flux réseau
    \item Zabbix déployé via Docker Compose
    \item Des agents Zabbix installés sur les clients
\end{itemize}

\newpage

% Architecture Réseau
\section{Architecture Réseau}

\subsection{Création du VPC}
La première étape consiste à créer un VPC (Virtual Private Cloud) isolé pour héberger notre infrastructure. Nous avons opté pour une configuration simple avec un sous-réseau public pour faciliter l'accès sans VPN.

\begin{figure}[H]
    \centering
    \includegraphics[width=0.8\textwidth]{captures/vpc_creation.png}
    \caption{Création du VPC 'TP-Zabbix-VPC' avec le bloc CIDR 10.0.0.0/16}
    \label{fig:vpc}
\end{figure}

\subsection{Configuration du sous-réseau et de la passerelle Internet}
Un sous-réseau public a été créé dans la zone de disponibilité us-east-1a. Une passerelle Internet a été attachée au VPC pour permettre la communication avec l'extérieur.

\begin{figure}[H]
    \centering
    \includegraphics[width=0.8\textwidth]{captures/subnet_config.png}
    \caption{Configuration du sous-réseau public avec le CIDR 10.0.1.0/24}
    \label{fig:subnet}
\end{figure}

\subsection{Table de routage}
La table de routage a été configurée pour diriger le trafic Internet (0.0.0.0/0) vers la passerelle Internet.

\begin{figure}[H]
    \centering
    \includegraphics[width=0.8\textwidth]{captures/route_table.png}
    \caption{Table de routage avec la route par défaut vers l'Internet Gateway}
    \label{fig:routetable}
\end{figure}

\subsection{Groupes de sécurité}
Deux groupes de sécurité ont été créés pour sécuriser les communications:

\subsubsection{Groupe de sécurité du serveur Zabbix}
Ce groupe contrôle l'accès au serveur Zabbix:
\begin{itemize}
    \item Port 22 (SSH): Accès administrateur
    \item Port 80/443 (HTTP/HTTPS): Interface web Zabbix
    \item Ports 10050/10051: Communications avec les agents Zabbix
\end{itemize}

\begin{figure}[H]
    \centering
    \includegraphics[width=0.8\textwidth]{captures/sg_server.png}
    \caption{Règles d'entrée du groupe de sécurité 'SG\_Serveur\_Zabbix'}
    \label{fig:sgserver}
\end{figure}

\subsubsection{Groupe de sécurité des clients}
Ce groupe sécurise les instances clientes:
\begin{itemize}
    \item Port 22 (SSH): Accès au client Linux
    \item Port 3389 (RDP): Accès au client Windows
    \item Ports 10050/10051: Restreints à l'IP privée du serveur Zabbix
\end{itemize}

\begin{figure}[H]
    \centering
    \includegraphics[width=0.8\textwidth]{captures/sg_clients.png}
    \caption{Règles d'entrée du groupe de sécurité 'SG\_Clients'}
    \label{fig:sgclients}
\end{figure}

\newpage

% Architecture des Instances EC2
\section{Déploiement des Instances EC2}

\subsection{Types d'instances choisies}
Conformément aux limitations du Learner Lab, nous avons déployé trois instances:

\begin{table}[H]
    \centering
    \begin{tabular}{|l|l|l|l|}
        \hline
        \textbf{Instance} & \textbf{Type} & \textbf{OS} & \textbf{Rôle} \\
        \hline
        Serveur Zabbix & t3.large & Ubuntu 22.04 & Serveur de monitoring \\
        Client Linux & t3.medium & Ubuntu 22.04 & Hôte monitoré Linux \\
        Client Windows & t3.large & Windows Server 2022 & Hôte monitoré Windows \\
        \hline
    \end{tabular}
    \caption{Configuration des instances EC2 déployées}
    \label{tab:instances}
\end{table}

\subsection{Instances en cours d'exécution}
Toutes les instances ont été lancées dans la région us-east-1 (N. Virginia) pour assurer la stabilité du Lab.

\begin{figure}[H]
    \centering
    \includegraphics[width=0.9\textwidth]{captures/ec2_instances.png}
    \caption{Récapitulatif des trois instances EC2 en cours d'exécution avec leurs IPs}
    \label{fig:ec2list}
\end{figure}

\subsection{Connexion au serveur Zabbix}
La connexion SSH à l'instance serveur a été établie avec succès.

\begin{figure}[H]
    \centering
    \includegraphics[width=0.8\textwidth]{captures/ssh_connection.png}
    \caption{Connexion SSH réussie à l'instance 'Serveur Zabbix'}
    \label{fig:ssh}
\end{figure}

\newpage

% Déploiement du Serveur Zabbix
\section{Déploiement du Serveur Zabbix en Mode Conteneurisé}

\subsection{Installation de Docker et Docker Compose}
Sur l'instance serveur, nous avons installé Docker et Docker Compose.

\begin{lstlisting}[caption=Installation de Docker et Docker Compose]
# Installation de Docker
sudo apt update && sudo apt upgrade -y
sudo apt install docker.io -y
sudo systemctl start docker
sudo systemctl enable docker

# Installation de Docker Compose
sudo apt install docker-compose -y
\end{lstlisting}

\subsection{Configuration du fichier docker-compose.yml}
Le fichier \texttt{docker-compose.yml} définit trois services: MySQL (base de données), Zabbix Server et l'interface web Zabbix.

\begin{lstlisting}[caption=Fichier docker-compose.yml, language=yaml]
version: '3.5'
services:
    mysql-server:
        image: mysql:8.0
        container_name: zabbix-mysql
        command: --character-set-server=utf8mb4 --collation-server=utf8mb4_unicode_ci
        volumes:
          - ./zabbix-mysql-data:/var/lib/mysql
        environment:
          MYSQL_DATABASE: 'zabbix'
          MYSQL_USER: 'zabbix'
          MYSQL_PASSWORD: 'zabbix_pwd'
          MYSQL_ROOT_PASSWORD: 'root_pwd'
        restart: always
        networks:
          - zbx_network

    zabbix-server:
        image: zabbix/zabbix-server-mysql:alpine-6.4-latest
        container_name: zabbix-server
        environment:
          DB_SERVER_HOST: 'mysql-server'
          MYSQL_DATABASE: 'zabbix'
          MYSQL_USER: 'zabbix'
          MYSQL_PASSWORD: 'zabbix_pwd'
          MYSQL_ROOT_PASSWORD: 'root_pwd'
        ports:
          - "10051:10051"
        depends_on:
          - mysql-server
        restart: always
        networks:
          - zbx_network

    zabbix-web:
        image: zabbix/zabbix-web-nginx-mysql:alpine-6.4-latest
        container_name: zabbix-web
        environment:
          ZBX_SERVER_HOST: 'zabbix-server'
          DB_SERVER_HOST: 'mysql-server'
          MYSQL_DATABASE: 'zabbix'
          MYSQL_USER: 'zabbix'
          MYSQL_PASSWORD: 'zabbix_pwd'
          MYSQL_ROOT_PASSWORD: 'root_pwd'
          PHP_TZ: 'Europe/Paris'
        ports:
          - "80:8080"
        depends_on:
          - mysql-server
          - zabbix-server
        restart: always
        networks:
          - zbx_network

networks:
    zbx_network:
        driver: bridge
\end{lstlisting}

\subsection{Lancement des conteneurs}
Les conteneurs ont été lancés avec Docker Compose.

\begin{figure}[H]
    \centering
    \includegraphics[width=0.8\textwidth]{captures/docker_compose_up.png}
    \caption{Exécution de la commande \texttt{sudo docker-compose up -d} sur le serveur}
    \label{fig:dockerup}
\end{figure}

\subsection{Vérification de l'état des conteneurs}
Les trois conteneurs sont bien en état 'Up', indiquant un fonctionnement correct.

\begin{figure}[H]
    \centering
    \includegraphics[width=0.8\textwidth]{captures/docker_compose_ps.png}
    \caption{Résultat de la commande \texttt{sudo docker-compose ps} montrant les conteneurs en état 'Up'}
    \label{fig:dockerps}
\end{figure}

\subsection{Accès à l'interface web Zabbix}
L'interface web est accessible via l'adresse IP publique du serveur.

\begin{figure}[H]
    \centering
    \includegraphics[width=0.8\textwidth]{captures/zabbix_login.png}
    \caption{Page de connexion à l'interface web de Zabbix}
    \label{fig:zabbixlogin}
\end{figure}

\begin{figure}[H]
    \centering
    \includegraphics[width=0.9\textwidth]{captures/zabbix_dashboard.png}
    \caption{Interface principale de Zabbix après connexion réussie}
    \label{fig:zabbixdashboard}
\end{figure}

\newpage

% Configuration des Clients
\section{Configuration des Clients (Agents Zabbix)}

\subsection{Configuration du client Linux}

\subsubsection{Installation de l'agent}
Sur l'instance client Linux, nous avons installé l'agent Zabbix.

\begin{lstlisting}[caption=Installation de l'agent sur Linux]
wget https://repo.zabbix.com/zabbix/6.4/ubuntu/pool/main/z/zabbix-release/zabbix-release_6.4-1+ubuntu22.04_all.deb
sudo dpkg -i zabbix-release_6.4-1+ubuntu22.04_all.deb
sudo apt update
sudo apt install zabbix-agent -y
\end{lstlisting}

\subsubsection{Configuration de l'agent}
Le fichier de configuration a été modifié pour pointer vers le serveur Zabbix.

\begin{figure}[H]
    \centering
    \includegraphics[width=0.9\textwidth]{captures/linux_agent_conf.png}
    \caption{Configuration du fichier /etc/zabbix/zabbix\_agentd.conf sur le client Linux}
    \label{fig:linuxconf}
\end{figure}

\subsubsection{Démarrage du service}
Le service a été démarré et configuré pour se lancer automatiquement.

\begin{figure}[H]
    \centering
    \includegraphics[width=0.8\textwidth]{captures/linux_agent_status.png}
    \caption{Démarrage du service zabbix-agent avec vérification de son statut}
    \label{fig:linuxstatus}
\end{figure}

\subsection{Configuration du client Windows}

\subsubsection{Installation de l'agent}
Sur l'instance Windows, l'agent a été installé via l'installateur MSI.

\begin{figure}[H]
    \centering
    \includegraphics[width=0.8\textwidth]{captures/windows_agent_install.png}
    \caption{Assistant d'installation de l'agent Zabbix pour Windows}
    \label{fig:wininstall}
\end{figure}

\subsubsection{Vérification du service}
Le service Windows a été vérifié dans le gestionnaire de services.

\begin{figure}[H]
    \centering
    \includegraphics[width=0.8\textwidth]{captures/windows_service.png}
    \caption{Vérification du service 'Zabbix Agent' dans le gestionnaire de services Windows}
    \label{fig:winservice}
\end{figure}

\newpage

% Monitoring et Tableaux de Bord
\section{Monitoring et Tableaux de Bord}

\subsection{Ajout des hôtes dans Zabbix}
Dans l'interface web de Zabbix, nous avons ajouté les deux clients comme nouveaux hôtes.

\begin{figure}[H]
    \centering
    \includegraphics[width=0.9\textwidth]{captures/add_host_linux.png}
    \caption{Formulaire de création de l'hôte pour le client Linux}
    \label{fig:addhostlinux}
\end{figure}

\subsection{Statut des hôtes}
Après quelques minutes, les deux hôtes apparaissent avec une disponibilité verte.

\begin{figure}[H]
    \centering
    \includegraphics[width=0.9\textwidth]{captures/hosts_green.png}
    \caption{Liste des hôtes dans Zabbix avec disponibilité VERTE (ZBX)}
    \label{fig:hostsgreen}
\end{figure}

\subsection{Visualisation des données}

\subsubsection{Client Linux}
Le graphique suivant montre l'évolution de la charge CPU du client Linux.

\begin{figure}[H]
    \centering
    \includegraphics[width=0.9\textwidth]{captures/linux_cpu_graph.png}
    \caption{Graphique de la charge CPU du client Linux sur les dernières minutes}
    \label{fig:linuxcpu}
\end{figure}

\subsubsection{Client Windows}
Le graphique suivant montre l'utilisation mémoire du client Windows.

\begin{figure}[H]
    \centering
    \includegraphics[width=0.9\textwidth]{captures/windows_memory_graph.png}
    \caption{Graphique de l'utilisation mémoire du client Windows}
    \label{fig:winmemory}
\end{figure}

\subsection{Données récentes}
La vue des données récentes permet de vérifier tous les indicateurs collectés.

\begin{figure}[H]
    \centering
    \includegraphics[width=0.9\textwidth]{captures/latest_data.png}
    \caption{Vue des données récentes pour le client Linux}
    \label{fig:latestdata}
\end{figure}

\newpage

% Conclusion
\section{Conclusion}

\subsection{Bilan du travail réalisé}
Ce travail pratique a permis de déployer avec succès une infrastructure de monitoring centralisée sur AWS. L'objectif principal a été atteint : la mise en place d'un serveur Zabbix conteneurisé capable de surveiller un parc hybride composé de machines Linux et Windows.

Les principales réalisations incluent:
\begin{itemize}
    \item Configuration complète d'un VPC avec sous-réseau public et groupes de sécurité
    \item Déploiement de trois instances EC2 avec les configurations appropriées
    \item Installation et configuration de Zabbix via Docker Compose
    \item Mise en place des agents sur les systèmes Linux et Windows
    \item Intégration réussie des hôtes dans l'interface Zabbix avec collecte de données
\end{itemize}

\subsection{Difficultés rencontrées et solutions}
Plusieurs difficultés ont été rencontrées durant ce TP:

\begin{itemize}
    \item \textbf{Limitations du Learner Lab}: Les instances s'arrêtent automatiquement après une période d'inactivité. La solution a été de redémarrer systématiquement les conteneurs Docker avec \texttt{docker-compose up -d} après chaque reprise du Lab.
    
    \item \textbf{Correspondance des noms d'hôtes}: Un décalage entre le nom configuré dans l'agent et celui déclaré dans Zabbix empêchait la communication. La solution a été de vérifier rigoureusement que les noms correspondent exactement.
    
    \item \textbf{Flux réseau bloqués}: Initialement, les agents ne parvenaient pas à communiquer avec le serveur. La résolution a consisté à vérifier et ajuster les règles des groupes de sécurité pour autoriser les ports Zabbix depuis l'IP privée du serveur.
\end{itemize}

\subsection{Améliorations possibles}
Pour aller plus loin, plusieurs améliorations pourraient être apportées:

\begin{itemize}
    \item Mise en place d'un Zabbix Proxy pour surveiller des réseaux distants
    \item Configuration d'alertes par email ou par d'autres canaux (Slack, Teams)
    \item Création de tableaux de bord personnalisés avec les métriques les plus critiques
    \item Surveillance d'autres services AWS via des templates spécifiques
    \item Automatisation du déploiement avec des outils comme Terraform ou CloudFormation
\end{itemize}

Ce TP a permis de démontrer la puissance et la flexibilité de la combinaison AWS/Docker/Zabbix pour la mise en place d'une solution de monitoring professionnelle, scalable et adaptée aux environnements hybrides.

\end{document}